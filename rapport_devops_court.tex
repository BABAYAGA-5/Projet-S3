\documentclass[11pt,a4paper]{article}

% Packages
\usepackage[utf8]{inputenc}
\usepackage[french]{babel}
\usepackage[T1]{fontenc}
\usepackage{geometry}
\usepackage{graphicx}
\usepackage{float}
\usepackage{hyperref}
\usepackage{listings}
\usepackage{xcolor}
\usepackage{titlesec}
\usepackage{fancyhdr}
\usepackage{tcolorbox}
\usepackage{enumitem}

% Page geometry
\geometry{margin=2cm}

% Code listing style
\definecolor{codegreen}{rgb}{0,0.6,0}
\definecolor{codegray}{rgb}{0.5,0.5,0.5}
\definecolor{codepurple}{rgb}{0.58,0,0.82}
\definecolor{backcolour}{rgb}{0.95,0.95,0.92}

\lstdefinestyle{mystyle}{
    backgroundcolor=\color{backcolour},
    commentstyle=\color{codegreen},
    keywordstyle=\color{magenta},
    numberstyle=\tiny\color{codegray},
    stringstyle=\color{codepurple},
    basicstyle=\ttfamily\footnotesize,
    breakatwhitespace=false,
    breaklines=true,
    captionpos=b,
    keepspaces=true,
    showspaces=false,
    showstringspaces=false,
    showtabs=false,
    tabsize=2
}
\lstset{style=mystyle}

% Header and footer
\pagestyle{fancy}
\fancyhf{}
\rhead{TP DevOps}
\lhead{Projet S3}
\rfoot{Page \thepage}

% Title formatting
\titleformat{\section}{\Large\bfseries\color{blue!70!black}}{\thesection}{1em}{}
\titleformat{\subsection}{\large\bfseries\color{blue!50!black}}{\thesubsection}{1em}{}

% Hyperref setup
\hypersetup{
    colorlinks=true,
    linkcolor=blue,
    filecolor=magenta,
    urlcolor=cyan,
    pdftitle={TP Global DevOps - Projet S3},
}

\begin{document}

% Title page
\begin{titlepage}
    \centering
    \vspace*{2cm}
    
    {\Huge\bfseries TP Global DevOps\par}
    \vspace{1cm}
    {\Large Pipeline CI/CD Complète\par}
    \vspace{1.5cm}
    
    {\Large\itshape Projet S3\par}
    {\large Application JEE de Gestion\par}
    
    \vspace{2cm}
    
    \begin{tcolorbox}[colback=blue!5!white,colframe=blue!75!black,title=Technologies]
        Git/GitHub • Jenkins • SonarQube • Docker • Kubernetes • Prometheus • Grafana
    \end{tcolorbox}
    
    \vfill
    
    {\large \url{https://github.com/BABAYAGA-5/Projet-S3}\par}
    \vspace{0.5cm}
    {\large Octobre 2025\par}
\end{titlepage}

% Table of contents
\tableofcontents
\newpage

% Introduction
\section{Introduction}

Le projet S3 est une application JEE de gestion comprenant plusieurs modules : annonces, clubs, réclamations, finances et authentification. L'objectif est de mettre en place une chaîne DevOps complète avec intégration continue, déploiement automatisé et supervision.

\textbf{Stack technique :} Java 21/JEE, Maven, Tomcat 10.1, JDBC, JSP

\newpage

% Étape 1 : GitHub
\section{Étape 1 : GitHub}

\subsection{Configuration}
\begin{itemize}
    \item Repository : \url{https://github.com/BABAYAGA-5/Projet-S3}
    \item Convention : GitFlow simplifiée (main + feature branches)
    \item Commits descriptifs en anglais
\end{itemize}

\begin{figure}[H]
    \centering
    \includegraphics[width=0.85\textwidth]{images/github_repository.png}
    \caption{Repository GitHub}
\end{figure}

\begin{figure}[H]
    \centering
    \includegraphics[width=0.85\textwidth]{images/git_history.png}
    \caption{Historique des commits}
\end{figure}

\newpage

% Étape 2 : Jenkins
\section{Étape 2 : Jenkins}

\subsection{Pipeline automatisé}
Le pipeline se déclenche automatiquement toutes les 5 minutes via \texttt{pollSCM('H/5 * * * *')}.

\textbf{9 étapes :}
\begin{enumerate}
    \item Checkout • Configuration Kubernetes • Build \& Test • Docker Build \& Push
    \item Deploy to Kubernetes • Deploy Monitoring • Setup Grafana • SonarQube • Verify
\end{enumerate}

\begin{tcolorbox}[colback=green!5!white,colframe=green!75!black,title=Résultats]
    \textbf{Status :} BUILD SUCCESS • \textbf{Durée :} 3-5 min • \textbf{Artefact :} Projet\_S3-1.0-SNAPSHOT.war
    
    \textbf{Image :} babayaga0/projet-s3:latest • \textbf{Pods :} 3 replicas running
\end{tcolorbox}

\textbf{URLs :} Application (30081) • Prometheus (30090) • Grafana (30300) • SonarQube (30900)

\begin{figure}[H]
    \centering
    \includegraphics[width=0.85\textwidth]{images/jenkins_pipeline.png}
    \caption{Pipeline Jenkins}
\end{figure}

\begin{figure}[H]
    \centering
    \includegraphics[width=0.85\textwidth]{images/jenkins_build_log.png}
    \caption{Log du build}
\end{figure}

\newpage

% Étape 3 : SonarQube
\section{Étape 3 : SonarQube}

\subsection{Analyse de qualité}
Commande : \texttt{.\textbackslash run-sonar-analysis.ps1}

\begin{tcolorbox}[colback=blue!5!white,colframe=blue!75!black,title=Résultats]
    \textbf{Fichiers :} 74 (39 Java, 32 JSP, 2 XML, 1 CSS) • \textbf{Durée :} 38s
\end{tcolorbox}

\textbf{Types d'anomalies :}
\begin{itemize}
    \item Code Smells : méthodes longues, complexité élevée, duplication
    \item Bugs : gestion exceptions, null pointers, ressources non fermées
    \item Sécurité : injection SQL, validation entrées, sessions non sécurisées
\end{itemize}

\begin{figure}[H]
    \centering
    \includegraphics[width=0.85\textwidth]{images/sonarqube_dashboard.png}
    \caption{Dashboard SonarQube}
\end{figure}

\begin{figure}[H]
    \centering
    \includegraphics[width=0.85\textwidth]{images/sonarqube_issues.png}
    \caption{Issues détectées}
\end{figure}

\newpage

% Étape 4 : Docker
\section{Étape 4 : Docker}

\subsection{Dockerfile}
Multi-stage build pour optimiser la taille de l'image :

\begin{lstlisting}[language=docker]
# Stage 1: Build avec Maven
FROM maven:3.9.9-eclipse-temurin-21 AS build
WORKDIR /app
COPY pom.xml .
COPY src ./src
RUN mvn -B -DskipTests package

# Stage 2: Runtime avec Tomcat
FROM tomcat:10.1-jdk21-temurin
LABEL maintainer="you@example.com"

# Supprimer les webapps par defaut
RUN rm -rf /usr/local/tomcat/webapps/*

# Deployer le WAR comme ROOT.war
COPY --from=build /app/target/*.war /usr/local/tomcat/webapps/ROOT.war

EXPOSE 8080
\end{lstlisting}

\textbf{Avantages :} Build en 2 étapes, image finale plus légère, pas besoin de Maven en prod

\subsection{Image publiée}
\begin{tcolorbox}[colback=blue!5!white,colframe=blue!75!black]
    \textbf{Nom :} babayaga0/projet-s3:latest • \textbf{Registry :} Docker Hub
\end{tcolorbox}

\textbf{Commandes :}
\begin{lstlisting}[language=bash]
docker build -t babayaga0/projet-s3:latest .
docker push babayaga0/projet-s3:latest
docker run -d -p 8080:8080 babayaga0/projet-s3:latest
\end{lstlisting}

\begin{figure}[H]
    \centering
    \includegraphics[width=0.85\textwidth]{images/docker_build.png}
    \caption{Build Docker}
\end{figure}

\begin{figure}[H]
    \centering
    \includegraphics[width=0.85\textwidth]{images/docker_hub.png}
    \caption{Docker Hub}
\end{figure}

\newpage

% Étape 5 : Kubernetes
\section{Étape 5 : Kubernetes}

\subsection{Déploiement}

\subsubsection{deployment.yaml}
\begin{lstlisting}[language=yaml]
apiVersion: apps/v1
kind: Deployment
metadata:
  name: projet-s3
spec:
  replicas: 2
  selector:
    matchLabels:
      app: projet-s3
  template:
    metadata:
      labels:
        app: projet-s3
    spec:
      containers:
      - name: projet-s3
        image: babayaga0/projet-s3:latest
        imagePullPolicy: Always
        ports:
        - containerPort: 8080
\end{lstlisting}

\subsubsection{service.yaml}
\begin{lstlisting}[language=yaml]
apiVersion: v1
kind: Service
metadata:
  name: projet-s3
spec:
  type: NodePort
  selector:
    app: projet-s3
  ports:
  - name: http
    port: 80
    targetPort: 8080
    nodePort: 30081
\end{lstlisting}

\textbf{Commandes de déploiement :}
\begin{lstlisting}[language=bash]
minikube start
kubectl apply -f k8s/
kubectl get all -l app=projet-s3
\end{lstlisting}

\begin{tcolorbox}[colback=green!5!white,colframe=green!75!black,title=Pods déployés]
    \textbf{Application :} 3 replicas projet-s3
    
    \textbf{Monitoring :} 1 Prometheus • 1 Grafana • 1 SonarQube
    
    \textbf{Total :} 6 pods actifs
\end{tcolorbox}

\textbf{Accès :} \texttt{minikube service projet-s3 --url} ou \url{http://localhost:30081}

\begin{figure}[H]
    \centering
    \includegraphics[width=0.85\textwidth]{images/kubectl_get_all.png}
    \caption{Ressources Kubernetes}
\end{figure}

\begin{figure}[H]
    \centering
    \includegraphics[width=0.85\textwidth]{images/app_deployed.png}
    \caption{Application déployée}
\end{figure}

\newpage

% Étape 6 : Prometheus & Grafana
\section{Étape 6 : Prometheus \& Grafana}

\subsection{Dashboard Grafana}
\textbf{Accès :} \url{http://localhost:30300} (admin/admin)

\textbf{10 panels :} Application Health • CPU Usage • Memory Usage • Network Traffic • Pod Status • Request Rate • Response Time • Error Rate • Database Connections • JVM Metrics

\subsection{KPI principale : Application Uptime}
\textbf{Justification :}
\begin{itemize}
    \item Impact direct sur l'expérience utilisateur
    \item Reflète la santé générale du système
    \item Métrique simple et compréhensible (UP/DOWN)
    \item Seuil clair pour alerting (< 99.9\%)
\end{itemize}

\subsection{Alertes configurées}

\textbf{Critiques (P1) :}
\begin{itemize}
    \item Application Down (> 1 min)
    \item Pod Crash Loop (> 5 restarts)
    \item High Error Rate (> 5\%)
\end{itemize}

\textbf{Importantes (P2) :}
\begin{itemize}
    \item High CPU (> 80\% pendant 10 min)
    \item High Memory (> 85\% pendant 10 min)
    \item Slow Response Time (p95 > 2s)
\end{itemize}

\begin{figure}[H]
    \centering
    \includegraphics[width=0.85\textwidth]{images/grafana_dashboard_overview.png}
    \caption{Dashboard Grafana}
\end{figure}

\begin{figure}[H]
    \centering
    \includegraphics[width=0.85\textwidth]{images/grafana_cpu_memory.png}
    \caption{Métriques CPU et Mémoire}
\end{figure}

\begin{figure}[H]
    \centering
    \includegraphics[width=0.85\textwidth]{images/prometheus_targets.png}
    \caption{Cibles Prometheus}
\end{figure}

\newpage

% Architecture
\section{Architecture globale}

\subsection{Flux CI/CD}
\begin{enumerate}
    \item Code poussé sur GitHub
    \item Jenkins déclenche le pipeline
    \item Build Maven + Tests
    \item Analyse SonarQube
    \item Build Docker + Push Docker Hub
    \item Déploiement Kubernetes
    \item Monitoring Prometheus/Grafana
\end{enumerate}

\begin{figure}[H]
    \centering
    \includegraphics[width=0.9\textwidth]{images/pipeline_architecture.png}
    \caption{Architecture CI/CD}
\end{figure}

\subsection{Stack technologique}
\begin{table}[H]
\centering
\begin{tabular}{|l|l|}
\hline
\textbf{Composant} & \textbf{Technologie} \\ \hline
Application & Java 21, JEE, Maven, Tomcat 10.1 \\ \hline
VCS & Git, GitHub \\ \hline
CI/CD & Jenkins \\ \hline
Qualité & SonarQube \\ \hline
Conteneurs & Docker, Docker Hub \\ \hline
Orchestration & Kubernetes (Minikube) \\ \hline
Monitoring & Prometheus, Grafana \\ \hline
\end{tabular}
\end{table}

\newpage

% Commandes utiles
\section{Commandes utiles}

\textbf{Git :}
\begin{lstlisting}[language=bash]
git clone https://github.com/BABAYAGA-5/Projet-S3
git log --oneline --graph
\end{lstlisting}

\textbf{Maven :}
\begin{lstlisting}[language=bash]
mvn clean package
mvn test
\end{lstlisting}

\textbf{Docker :}
\begin{lstlisting}[language=bash]
docker build -t babayaga0/projet-s3:latest .
docker push babayaga0/projet-s3:latest
docker run -d -p 8080:8080 babayaga0/projet-s3:latest
\end{lstlisting}

\textbf{Kubernetes :}
\begin{lstlisting}[language=bash]
# Deploiement
kubectl apply -f k8s/deployment.yaml
kubectl apply -f k8s/service.yaml
kubectl apply -f k8s/monitoring/

# Verification
kubectl get all
kubectl get pods -l app=projet-s3
kubectl get svc
kubectl describe deployment projet-s3

# Logs et debugging
kubectl logs <pod-name>
kubectl logs -f <pod-name>  # Follow logs
kubectl exec -it <pod-name> -- /bin/bash

# Port forwarding
kubectl port-forward svc/projet-s3 8080:80
kubectl port-forward svc/grafana 3000:3000
\end{lstlisting}

\textbf{Minikube :}
\begin{lstlisting}[language=bash]
# Demarrage et configuration
minikube start
minikube status
minikube update-context

# Services
minikube service list
minikube service projet-s3 --url
minikube service grafana --url

# Monitoring
minikube dashboard
minikube ssh  # SSH dans le node

# Nettoyage
minikube stop
minikube delete
\end{lstlisting}

\newpage

% Conclusion
\section{Conclusion}

\subsection{Objectifs atteints}
\begin{itemize}
    \item[$\checkmark$] Gestion du code source avec Git/GitHub
    \item[$\checkmark$] Pipeline CI/CD automatisée avec Jenkins
    \item[$\checkmark$] Analyse qualité avec SonarQube
    \item[$\checkmark$] Containerisation avec Docker
    \item[$\checkmark$] Déploiement Kubernetes
    \item[$\checkmark$] Monitoring Prometheus/Grafana
\end{itemize}

\subsection{Résultats}
\begin{tcolorbox}[colback=blue!5!white,colframe=blue!75!black,title=Bilan]
    \textbf{Pipeline :} 9 étapes automatisées • \textbf{Pods :} 6 actifs (3 app + 3 monitoring)
    
    \textbf{Monitoring :} 10 panels métriques • \textbf{Alertes :} 9 configurées (3 niveaux)
    
    \textbf{Disponibilité :} Application stable et accessible
\end{tcolorbox}

\subsection{Améliorations possibles}
\begin{itemize}
    \item Tests d'intégration et E2E
    \item Webhooks GitHub pour déclenchement instantané
    \item Environnements multiples (dev, staging, prod)
    \item Tests de sécurité (OWASP)
    \item Auto-scaling Kubernetes
    \item Blue/Green ou Canary deployments
\end{itemize}

\subsection{Bilan}
Ce projet démontre la mise en œuvre complète des pratiques DevOps modernes : automatisation du cycle de vie de l'application, de la compilation au déploiement, avec monitoring en temps réel. L'infrastructure est entièrement opérationnelle et supervisée.

\newpage

% Annexes
\section{Annexes}

\subsection{Liens}
\begin{itemize}
    \item GitHub : \url{https://github.com/BABAYAGA-5/Projet-S3}
    \item Docker Hub : \url{https://hub.docker.com/r/babayaga0/projet-s3}
\end{itemize}

\subsection{Jenkinsfile (extrait)}
\begin{lstlisting}[language=groovy]
pipeline {
  agent any
  
  environment {
    DOCKER_IMAGE = 'babayaga0/projet-s3'
    DOCKER_TAG = 'latest'
    KUBECONFIG = 'C:\\ProgramData\\Jenkins\\.kube\\config'
  }
  
  triggers {
    pollSCM('H/5 * * * *')  // Poll every 5 minutes
  }
  
  stages {
    stage('Checkout') { ... }
    stage('Configure Kubernetes') { ... }
    stage('Build & Test') {
      steps {
        bat 'mvn clean package'
      }
    }
    stage('Docker Build & Push') {
      steps {
        bat "docker build -t ${DOCKER_IMAGE}:${DOCKER_TAG} ."
        bat "docker push ${DOCKER_IMAGE}:${DOCKER_TAG}"
      }
    }
    stage('Deploy to Kubernetes') {
      steps {
        bat 'kubectl apply -f k8s/'
      }
    }
  }
}
\end{lstlisting}

\subsection{Structure projet}
\begin{lstlisting}
Projet-S3/
├── Dockerfile                      # Multi-stage build
├── Jenkinsfile                     # Pipeline CI/CD
├── pom.xml                         # Config Maven
├── sonar-project.properties        # Config SonarQube
├── run-sonar-analysis.ps1          # Script analyse
├── k8s/
│   ├── deployment.yaml             # 2 replicas
│   ├── service.yaml                # NodePort 30081
│   ├── ingress.yaml                # Ingress (optionnel)
│   └── monitoring/
│       ├── prometheus-config.yaml
│       ├── prometheus-deployment.yaml
│       ├── grafana-deployment.yaml
│       └── sonarqube-deployment.yaml
├── src/main/
│   ├── JAVA/                       # Code source
│   │   ├── Bean/
│   │   ├── DAO/
│   │   └── servlets/
│   ├── resources/
│   └── webapp/                     # JSP, CSS
├── docs/
│   └── projet-s3-dashboard.json    # Dashboard Grafana
└── target/                         # Build artifacts
\end{lstlisting}

\subsection{sonar-project.properties}
\begin{lstlisting}
sonar.projectKey=JAVA_Projet_S3
sonar.projectName=Projet S3
sonar.sources=src/main/JAVA
sonar.java.binaries=target/classes
sonar.sourceEncoding=UTF-8
\end{lstlisting}

\subsection{Monitoring - prometheus-config.yaml (extrait)}
\begin{lstlisting}[language=yaml]
apiVersion: v1
kind: ConfigMap
metadata:
  name: prometheus-config
data:
  prometheus.yml: |
    global:
      scrape_interval: 15s
    
    scrape_configs:
      - job_name: 'kubernetes-pods'
        kubernetes_sd_configs:
        - role: pod
        
      - job_name: 'projet-s3'
        static_configs:
        - targets: ['projet-s3:80']
\end{lstlisting}

\end{document}
